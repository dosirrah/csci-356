\documentclass{article}
\usepackage{amsmath}
\usepackage{amssymb}
\usepackage{booktabs} % For better looking tables
\usepackage{graphicx}
\usepackage[utf8]{inputenc} % Allow input to be UTF-8
\usepackage{mdframed}   % used in defining graybox
\usepackage{enumitem}   % used so we can say label=(\alph*) for enumerations.
% to control margins.
\usepackage[left=1.4in, right=1.4in, top=1in, bottom=1in]{geometry}

\newmdenv[backgroundcolor=lightgray,innerleftmargin=10pt,innerrightmargin=10pt,innertopmargin=10pt,innerbottommargin=10pt]{graybox}


% This is written in latex.  It is more powerful than markdown by far,
% but it isn't quite as easy to read.


\title{CSCI 356 Answers Homework 5: Bonus}
\date{Due: Wednesday, November 29, 11:59 PM}

\begin{document}

\maketitle % This command creates the title using the information provided above


% Define custom light gray color
\definecolor{lightgray}{gray}{0.9}

\section{Problem 1}

(10 points divided 2, 2, 3, 3 bewteen (a) through (d) respectively)

Analyze the time complexity of each of the following code snippets.
Use the example problems as guide as to what constitutes a sufficient answer.

\begin{enumerate}[label=(\alph*)]

\item 
\begin{verbatim}
  x *= 2
\end{verbatim}

\subsection*{Answer 1(a)}
\begin{verbatim}
  x *= 2  # 1 step
\end{verbatim}

\begin{equation*}
  T = 2
\end{equation*}

\begin{equation*}
  \boxed{T = O(1)}
\end{equation*}
      
\item
\begin{verbatim}
  x = 5
  y = x + 1
\end{verbatim}

\subsection*{Answer 1(b)}
\begin{verbatim}
  x = 5       # 1 step
  y = x + 1   # 1 step (add) + 1 step (assign)
\end{verbatim}

\begin{equation*}
  T = 3
\end{equation*}

\begin{equation*}
  \boxed{T = O(1)}
\end{equation*}

\item
\begin{verbatim}
  x = 0
  for i in range(n):
      i *= 2
\end{verbatim}

\subsection*{Answer 1(c)}
\begin{verbatim}
  x = 0                # (1)  1 step
  for i in range(n):   # (2)  n * (3)
      i *= 2           # (3)  1 step
\end{verbatim}

\begin{equation*}
  T = 1 + n * 1 = 1 + n
\end{equation*}

\begin{equation*}
  \boxed{T = O(n)}
\end{equation*}



\item
\begin{verbatim}
  def g(y: int) -> None:
    return y * 2

  def f(x: int) -> None:
    for j in range(x):
      _ = g(x)     # _ means "ignore this variable."

  for i in range(n):
    f(n)
\end{verbatim}

\subsection*{Answer 1(d)}

\begin{verbatim}
  def g(y: int) -> None:  
    return y * 2         # (1) 1 mult + 1 pop activation record

  def f(x: int) -> None:
    for j in range(x):   # (2) n*(3) because x=n
      _ = g(x)           # (3) T_g + 1 + 1 push activation record

  for i in range(n):     # (4) n * (5)
    f(n)                 # (5) T_f + 1 push activation record
\end{verbatim}

Let $T_g$ denote the time to execute function \verb|g(y)|.

Let $T_f$ denote the time to execute function \verb|f(n)|.

\begin{eqnarray}
  T(n) &=& n \cdot T_f(n) + 1  \label{eq:1dT}  \\
  T_f(n) &=& n \cdot (T_g(n) + 2)  \label{eq:1dTf} \\
  T_g(n) &=& 2 \label{eq:1dTg} \\
\end{eqnarray}

Substituting Equation \ref{eq:1dTg} into Equation \ref{eq:1dTf} yields

\begin{equation*}
  T_f(n) = 4n
\end{equation*}

We then substitute this in to Equation \ref{eq:1dT} to get

\begin{equation*}
  T(n) = 4n^2 + 1
\end{equation*}

\begin{equation*}
  \boxed{T(n) = O(n^2)}
\end{equation*}

\end{enumerate}



\section{Problem 2}

(20 points divided 2, 2, 2, 3, 3, 3, 5 bewteen (a)-(g) respectively)

Now we introduce Python's built-in list.  For extra credit, when
specifying big-O notation include the word ``amortized'' when the cost
is known to have an amortized cost that exceeds the $g(n)$ when
stating that a function $f(n) = O(g(n))$.  For example, appending to a
Python list is amortized $O(1)$.  When not considering amortized
costs, the actual worst case performance for an append to a python
list is $O(n)$ because the need to copy all $n$ items when the
allocated array holding the $n$ items is full.

When a problem has $n$ or $m$, but these variables have no assigned value
within the code snippet, assume that they have been set before this code
snippet.

Assume the following has been executed near the top of the file containing
these code snippets:

\begin{verbatim}
  from random import randint
  from bisect import bisect_left
\end{verbatim}

\begin{enumerate}[label=(\alph*)]

\item (2 points + 1 extra for amortized)
  
\begin{verbatim}
  x = []
  x.append(5)
\end{verbatim}

\subsection*{Answer 2(a)}

\begin{verbatim}
  x = []         # 1 step
  x.append(5)    # amortized O(1)
\end{verbatim}

\begin{equation*}
  T = 1 + \text{ amortized } O(1)
\end{equation*}

Amortized $O(1)$ is bigger than 1 so we can ignore the 1.

\begin{equation*}
  \boxed{T = \text{ amortized } O(1)}
\end{equation*}


\item (2 points)

\begin{verbatim}
  x = []
  for i in range(n):
      x.append(i)
\end{verbatim}

\subsection*{Answer 2(b)}

\begin{verbatim}
  x = []              # (1)  1 step
  for i in range(n):  # (2)  n * (3) 
      x.append(i)     # (3)  amortized O(1)
\end{verbatim}

\begin{equation*}
  T(n) = 1 + n \cdot \text{ amortized } O(1) 
\end{equation*}


Amortized $O(1)$ means a cost was amortized over many operations to
obtain an average of $O(1)$.  In the case of \verb|append|, the
amortization is across $n$ append operations.  Amortized $O(1) = O(n)
/ n$.  Thus, $n \cdot \text{ amortized } O(1) = O(n)$.

\begin{equation*}
  \boxed{T(n) = O(n)}
\end{equation*}


\item (2 points)

\begin{verbatim}
  x = [randint(0, 1000) for x in range(n)]
\end{verbatim}

\subsection*{Answer 2(c)}

\begin{verbatim}
  x = [randint(0, 1000) for x in range(n)]  # n * O(1)
\end{verbatim}

\begin{equation}
  \boxed{T(n) = O(n)}
\end{equation}

\item (3 points) express the time complexity as a function of $n$ and $m$.
  $x$ is an array containing $n$ integers. $m < n$. (3 points)

\begin{verbatim}
  for _ in range(m):
    i = randint(0, len(x)-1)
    v = x[i]
    x.remove(v)
\end{verbatim}

\subsection*{Answer 2(d)}

\begin{verbatim}
  for _ in range(m):         # (1)  m * (2 to 4) 
    i = randint(0, len(x)-1) # (2)  O(1)
    v = x[i]                 # (3)  O(1)
    x.remove(v)              # (4)  O(n)
\end{verbatim}

\begin{equation*}
  T(n) = m \cdot ( O(1) + O(1) + O(n)
\end{equation*}

\begin{equation*}
  \boxed{T(n) = O(nm)}
\end{equation*}

\item (3 points)

\begin{verbatim}
  x = []
  for _ in range(n):
    i = randint(0, len(x)-1)
    x.insert(i, randint(0, 100000))
\end{verbatim}

\subsection*{Answer 2(e)}
    
\begin{verbatim}
  x = []                        # (1) 1 step
  for _ in range(n):            # (2) n * (3 to 4)
    i = randint(0, len(x)-1)    # (3) O(1) rand + O(1) len
    x.insert(i, randint(0, 100000)) # (4) O(1) + O(n) insert
\end{verbatim}

\begin{equation*}
  T(n) = 1 + n \cdot (O(1) + O(1) + O(1) + O(n)) 
\end{equation*}

\begin{equation*}
  \boxed{T(n) = O(n^2)}
\end{equation*}


\item (3 points)
\begin{verbatim}
   n = len(x)
   for i in range(n):
      i = randint(0, len(x)-1)
      x[i] = x[i] + 1
\end{verbatim}

\subsection*{Answer 2(f)}

\begin{verbatim}
   n = len(x)                   # (1) O(1) len + 1 assign
   for i in range(n):           # (2) n * (3 to 4)
      i = randint(0, len(x)-1)  # (3) O(1) len + O(1) randint + 1 assign
      x[i] = x[i] + 1           # (4) O(1) getitem + 1 + 1 + O(1) setitem
\end{verbatim}

\begin{equation*}
  T(n) = O(1) + 1 + n \cdot (O(1) + O(1) + 1 + O(1) + 1 + 1 + O(1))
\end{equation*}

\begin{equation*}
  \boxed{T(n) = O(n)}
\end{equation*}


\item $n$ and $k$ are integers.  $k < n$.  Express the time complexity as
  a function of $k$ and $n$. (5 points)
  
\begin{verbatim}
    x = []
    i = 0
    for _ in range(n):
        i += randint(0, 10)
        x.append(i)

    print(f"inserted {n} items into list x in increasing order.  Max value is {i}.")
    found = 0
    for _ in range(k):
        y = randint(0, i)
        j = bisect_left(x, y)
        if j < len(x) and x[j] == y:
            found += 1
\end{verbatim}

\subsection{Answer 2(g)}

\begin{verbatim}
    x = []                           # (1)  1 assign
    i = 0                            # (2)  1 assign
    for _ in range(n):               # (3)  n * (4 to 5)
        i += randint(0, 10)          # (4)  O(1) randint + 1 plusassign
        x.append(i)                  # (5)  amortized O(1)
                                     
    print(f"inserted {n} ...")       # (6)  1 print
    found = 0                        # (7)  1 assign
    for _ in range(k):               # (8)  k * (9 to 12)
        y = randint(0, i)            # (9)  1 randint
        j = bisect_left(x, y)        # (10) O(log n) bisect + 1 assign
        if j < len(x) and x[j] == y: # (11) 1 less + O(1) len + 1 + 1
            found += 1               # (12) 1 plusassign
\end{verbatim}

Let $T_1(n)$ refer to the time complexity for executing lines (1) to (5).

Let $T_2(n)$ refer to the time complexity for executing lines (6) to (12).

\begin{eqnarray*}
  T(n) &=& T_1(n) + T_2(n)1 \\
  T_1(n) &=& 1 + n \cdot ( O(1) + 1 + \text{ amortized } O(1)) \\
  T_1(n) &=& O(n) \label{eq:2gT1} \\
  T_2(n) &=& 1 + 1 + k \cdot (1 + O(\log n) + 1 + 1 + O(1) + 1 + 1) \\
  T_2(n) &=& O(k \log n) 
\end{eqnarray*}

In Equation \ref{eq:2gT1} we drop the ``amortized'' because the
amortization of the $n$ number of amortized $O(1)$ operations is
amortized (spread) over the $n$ operations resulting in a total worst
case performance of $O(n)$.

\begin{equation*}
  \boxed{T(n) = O(n + k \log n)}
\end{equation*}

We cannot simplify it further because $k$ can be any value $1 \leq k
\leq n$.  If $k$ is a constant (i.e., not varying with $n$) then
$O(k\log n)$ is $O(\log n)$. However, when $k=O(n)$, $O(k \log n)$
becomes $O(n \log n)$ which is greater than $O(n)$.

\end{enumerate}


\section{Problem 3} (10 points divided 3, 3, 4)

Now we add dicts.  Analyze the following code snippets. Use same
assumptions as in problem 3.  dicts

\begin{enumerate}[label=(\alph*)]

\item (3 points + 1 extra for ``average'')
  
\begin{verbatim}
  d = {}
  for i in range(n):
    j = randint(0, len(d)-1)
    d[i] = j
\end{verbatim}

\subsection*{Answer 3(a)}

\begin{verbatim}
  d = {}                      # (1) 1 assign
  for i in range(n):          # (2) n * (3 to 4)
    j = randint(0, len(d)-1)  # (3) 1 rand + 1 len + 1 minus
    d[i] = j                  # (4) 1 assign + avg O(1) setitem
\end{verbatim}

\begin{eqnarray*}
  T(n) &=& 1 + n \cdot (1 + 1 + 1 + 1 + \text{ average } O(1) \\
       &=& n \cdot (4 + \text{ average } O(1))
\end{eqnarray*}

\begin{equation*}
  \boxed{T(n) = \text{ average } O(n)} 
\end{equation*}

\item In this problem the first line contains a list comprehension
  that creates a list of $n$ key-value pairs.  This step takes $O(n)$.
  Express the time complexity in terms of $n$ and $m$. (3 points + 1
  extra for ``average'')
  
\begin{verbatim}
  kv = [(randint(0, 10000), randint(0, 10000)) for _ in range(n]
  d = dict(kv)
  x = 0
  for i in range(m):
     i = randint(0, n-1)
     x += d[kv[i]]   # OOPS!  Should be d[kv[i][0]]
\end{verbatim}

\subsection*{Answer 3(b)}

\begin{verbatim}
  kv = [(randint(0, 10000), randint(0, 10000)) for _ in range(n] # (1)
  d = dict(kv)           # (2) average O(len(kv)) = average O(n)
  x = 0                  # (3) 1 assign
  for i in range(m):     # (4) m * (5 to 6)
     i = randint(0, n-1) # (5) 1 minus + O(1) randint + 1 assign
     x += d[kv[i][0]]    # (6) 1 getitem + average O(1) setitem + 1
\end{verbatim}

The first line requires $O(n)$ time to populate the key-value pairs.
Let $T_{for}$ be the time to execute the for loop in lines (4)-(6).

\begin{eqnarray*}
  T(n) &=& O(n) + \text{ average } O(n) + 1 + T_{for}(n) \\
  T_{for}(n) &=& m \cdot (1 + O(1) + 1 + 1 + \text{ average } O(1) + 1) \\
  &=& \text{ average } O(m) \\
  T(n) &=& O(n) + \text{ average } O(n) + \text{ average } O(m)
\end{eqnarray*}

For any given $n$, ``average $O(n)$'' could exceed $O(n)$, as such
we can treat $O(n)$ as smaller than ``average $O(n)$'' and thus we
drop the $O(n)$ but keep ``average $O(n)$.''

\begin{equation*}
  \boxed{T(n) = \text{ average } O(n + m)}
\end{equation*}



\item Express time complexity in terms of $n$ and $m$. (4 points + 1 extra for ``average'')
  
\begin{verbatim}
  k = 0
  kv = []
  for _ in range(n):
      k += randint(0, 10)
      v = randint(0, 100000)
      kv.append( (k, v) )

  d = dict(kv)
  for i in range(m):
     k, v = kv.pop()
     del d[k]
\end{verbatim}

\subsection*{Answer 3(c)}

\begin{verbatim}
  k = 0                       # (1) 1 step
  kv = []                     # (2) 1 assign
  for _ in range(n):          # (3) n * (4 to 6)
      k += randint(0, 10)     # (4) O(1) randint + 1 plusassign
      v = randint(0, 100000)  # (5) O(1) randint + 1 assign
      kv.append( (k, v) )     # (6) amortized O(1) append

  d = dict(kv)                # (7) average O(n) copy.
  for i in range(m):          # (8) m * (9 to 11)
     k, v = kv.pop()          # (9) amortized O(1) pop + 1 assign
     del d[k]                 # (10) average O(1) delete.
\end{verbatim}

Let $T_1(n)$ refer to the time complexity to execute steps (1)-(6).

Let $T_2(n)$ refer to the time complexity to execute steps (7)-(10). 

\begin{eqnarray}
  T(n) &=& T_1(n) + T_2(n) \\
  T_1(n) &=& 2 + n \cdot (O(1) + 1 + O(1) + 1 + \text{ amortized } O(1)) \\
  &=& n \cdot \text{ amortized } O(1) \\
  &=& O(n) \label{eq:3cT1} \\
  T_2(n) &=& \text{ average } O(n)
  + m \cdot (\text{ amortized } O(1) + 1 + \text{ average } O(1)) \\
  &=& \text{ average } O(n) + \text{ amortized } O(m) +
  \text{ average } O(m) \label{eq:3cT2} 
\end{eqnarray}

In Equation \ref{eq:3cT1} the qualifier ``amortized'' is dropped,
because the ``amortized'' part of the $O(1)$ operations has been
amortized over the $n$ operations resulting in a worst-case
performance of $O(n)$.  NOTE: The amortized $O(1)$ comes from taking
$O(n)$ and dividing by $n$.  Thus,

\begin{eqnarray*}
  \text{amortized } O(1) &=& \frac{O(n)}{n} \\
  n \cdot \text{ amortized } O(1) &=& O(n)
\end{eqnarray*}

\begin{eqnarray*}
  T(n) &=& O(n) + \text{ average } O(n) + \text{ amortized } O(m) +
  \text{ average } O(m) \\
       &=& \text{ average } O(m + n) + \text{ amortized } O(m)
\end{eqnarray*}

Because ``average'' is a weaker guarantee than ``amortized'' we drop
amortized.

\begin{equation}
  \boxed{T(n) = \text{ average } O(m + n)}
\end{equation}

\end{enumerate}

\section{Problem 4} (10 points)

Now we consider queues.

We implement a queue using a list as shown in Figure~\ref{fig:queue}.
What is the time complexity of the following code:

\begin{figure}
\begin{verbatim}
  class Queue:
    """
    A Queue implemented by wrapping a Python list.

    """

    def __init__(self):
        self.items = []

    def is_empty(self) -> bool:
        """
        :return: whether the Queue is empty.
        """
        return self.items == []

    def enqueue(self, item) -> None:
        """
        Enqueue the passed item to the end of the queue.

        :param item: item to be enqueued.
        :return: None
        """
        self.items.insert(0,item)

    def dequeue(self) -> object:
        """
        Remove the first item in the queue.
        :return: the item removed from the front of the queue.
        :raises IndexError: if Queue empty.
        """
        return self.items.pop()

    def __len__(self):
        """
        :return: the number of items in the Queue.
        """
        return len(self.items)

\end{verbatim}
\caption{Implementation of a Queue provided for HW3 p4.}\label{fig:queue}

\end{figure}

\begin{enumerate}[label=(\alph*)]

\item (3 points)
\begin{verbatim}
  q = Queue()
  for i in range(n):
      q.enqueue(randint(0, 1000))
  for i in range(n):
      _ = q.dequeue()
\end{verbatim}

\subsection*{Answer 4(a)}

\begin{verbatim}
  q = Queue()                      # (1) O(1)
  for i in range(n):               # (2) n * (3)
      q.enqueue(randint(0, 1000))  # (3) 1 randint + O(n)
  for i in range(n):               # (4) n * (5)
      _ = q.dequeue()              # (5) O(1)
\end{verbatim}

\begin{equation*}
  T(n) = O(1) + n \cdot (1 + O(n)) + n \cdot O(1)
\end{equation*}

\begin{equation*}
  \boxed{T(n) = O(n^2)}
\end{equation*}

\item (3 points)

\begin{verbatim}
  q = Queue()
  for i in range(n):
      q.enqueue(randint(0, 1000))
      _ = q.dequeue()
\end{verbatim}

\subsection*{Answer 4(b)}

\begin{verbatim}
  q = Queue()                      # (1) O(1)
  for i in range(n):               # (2) n * (3 to 4)
      q.enqueue(randint(0, 1000))  # (3) 1 randint + O(1)
      _ = q.dequeue()              # (4) O(1)
\end{verbatim}

Note that in steps (3) and (4), the queue never grows beyond
1 element at any given time.  The enqueue always occurs on an empty
queue, as such there are no elements to shift right causing
the enqueue to become an $O(n)$ operation.

\begin{equation*}
  T(n) = O(1) + n \cdot (1 + O(1) + O(1)))
\end{equation*}

\begin{equation*}
  \boxed{T(n) = O(n)}
\end{equation*}

\vspace{0.2in}

Now consider a Queue implemented usinq a singly-linked list as shown
in Figures \ref{fig:sll_queue} and \ref{fig:sll}.  These
implementations can also be found in the repository at

\begin{verbatim}
    csci-356/lecture15to17/sll_queue.py
\end{verbatim}

and

\begin{verbatim}
    csci-356/lecture15to17/singly_linked_list.py
\end{verbatim}

\item (4 points)
\begin{verbatim}
  q = SLLQueue()
  for i in range(n):
      q.enqueue(randint(0, 1000))
  for i in range(n):
      _ = q.dequeue()
\end{verbatim}

\begin{figure}
\begin{verbatim}
from singly_linked_list import SinglyLinkedList

class SLLQueue:
    """Implementation of a Queue based on Python's list."""

    def __init__(self):
        self._items = SinglyLinkedList()

    def is_empty(self):
        return len(self._items) == 0

    def enqueue(self, item):
        # O(n) operation.
        self._items.push_back(item)

    def dequeue(self):
        return self._items.pop_front()

    def __len__(self):
        return len(self._items)

\end{verbatim}

\caption{Queue implementation based on a singly-linked list.  This was
  covered somewhere in lectures 15 to 17, and can be found in the
  directory csci-356/lecture15to17.}\label{fig:sll_queue}
\end{figure}



\begin{figure}
\begin{verbatim}
  class SinglyLinkedList:
    """Singly-linked list."""
    class _Node:
        def __init__(self, item: object):
            self._next = None
            self._item = item
     
    def __init__(self):
        """Constructor of the SinglyLinkedList."""
        self._front = None
        self._end = None
        self._n = 0

    def push_back(self, x) -> None:
        """Pushes the item x to the rear of the linked list."""
        if self._front is None:
            assert self._end is None
            node = SinglyLinkedList._Node(x)
            self._front = self._end = node
            self._n = 1
        else:
            node = SinglyLinkedList._Node(x)
            self._end._next = node
            self._end = node
            self._n += 1

    def __len__(self) -> int:
        return self._n

    def pop_front(self) -> object:
        if self._n == 0:
            raise IndexError("pop from empty linked list.")

        node = self._front
        self._front = node._next
        if self._front is None:
            self._end = None

        item = noded._item
        self._n -= 1
        del node
        return item
\end{verbatim}
\caption{Implementation of singly-linked list. This also appears in
  the repository in csci-356/lecture15to17. I omitted some of the comments
  to fit the implementation on a single page.}\label{fig:sll}

\end{figure}

\subsection*{Answer 4(c)}
\begin{verbatim}
  q = SLLQueue()                  # (1) O(1)
  for i in range(n):              # (2) n * (3)
      q.enqueue(randint(0, 1000)) # (3) O(1) randint + O(1)
  for i in range(n):              # (4) n * (5)
      _ = q.dequeue()             # (5) O(1)
\end{verbatim}

\begin{equation*}
  T(n) = O(1) + n \cdot (O(1) + O(1)) + n \cdot O(1)
\end{equation*}

\begin{equation*}
  \boxed{T(n) = O(n)}
\end{equation*}

\end{enumerate}

\section{Problem 5}

(10 points divided 5, 5).

Now we consider recursion.

\begin{enumerate}[label=(\alph*)]
\item  What is the final output and what is the time complexity as a
  function of $n$. Although $n$ is assigned the constant 512, express
  the time complexity as if $n$ is not a constant.  $n$ is set to 512
  simply to allow you to specify the final output. (5 points +1 extra for ``amortized'')
  
\begin{verbatim}
    n = 512
    def f(n):
        if n <= 1:
           return [n]
        return f(n/2).extend(n)  # OOPS! should be append(n).  

    print(f(n))
\end{verbatim}

\subsection*{Answer 5(a)}

\begin{verbatim}
    n = 512                     # (1) 1 assign
    def f(n):          
        if n <= 1:              # (2) 1 compare
           return [n]           # (3) O(1) + 1 pop activation record.
        return f(n/2).append(n) # (4) T_f(n/2) + amortized O(1) + 1 

    print(f(n))                 # (5) T_f(n) + 1 push + 1 print.
\end{verbatim}


\begin{eqnarray*}
  T(n) &=& 1 + T_f(n) + 1 \\
  T_f(1) &=& 1 + 1 + O(1) \\
  T_f(1) &=& O(1) \\
  T_f(n) &=& 1 + T_f(n/2) + \text{ amortized } O(1) 
\end{eqnarray*}

$f(n)$ gets called $\log_2 n$ times before terminating.

\begin{equation*}
  T(n) = 2 + (\log_2 n) \cdot (1 + O(1) + \text{ amortized } O(1)
\end{equation*}

\begin{equation*}
  \boxed{T(n) = \text{ amortized } \log n}
\end{equation*}

Output:

\begin{verbatim}
  [1, 2, 4, 8, 16, 32, 64, 128, 256, 512]
\end{verbatim}


\item What is the final output and what is the time complexity of
  \verb|smallest_factor()| as a function of $n$.
  
\begin{verbatim}
def smallest_factor(n, i=2):
    if i * i > n:
        return n
    
    # If i is a factor of n, return i
    if n % i == 0:
        return i
    
    return smallestFactor(n, i + 1)

# Example usage
print(smallest_factor(49)) 
print(smallest_factor(37))
\end{verbatim}

\subsection*{Answer 5(b)}

\begin{verbatim}
def smallest_factor(n, i=2):
    if i * i > n:                   # (1) 1 comp + 1 mult
        return n                    # (2) 1 pop act rec
    
    # If i is a factor of n, return i 
    if n % i == 0:                  # (3) 1 mod + 1 comp
        return i                    # (4) 1 pop act rec
    
    return smallestFactor(n, i + 1) # (5) T_{sf}(n,i+1) + 1 push 

# Example usage
print(smallest_factor(49))          # (6) T_{sf} + 1 print
print(smallest_factor(37))          # (7) T_{sf} + 1 print
\end{verbatim}

\begin{eqnarray*}
  T(n,2) &=& 2 \cdot T_{sf}(n, 2) + 2\\
  T_{sf}(n,2) &=& 1 + 1 + 1 + 1 + T_{sf}(n, 3) + 1 \\
  T_{sf}(n,2) &=& 5 + T_{sf}(n, 3) \\
  T_{sf}(n,i) &=& 5 + T_{sf}(n, i+1) \\
  T_{sf}(n, \sqrt{n}) &=& 3
\end{eqnarray*}

$T_{sf}(n, \sqrt{n})$ is the terminating condition.

With each recursion, $i$ increases by 1 until it reaches the $\sqrt{n}$.
Since starts at 2, $T_{sf}(n, 2)$ becomes

\begin{eqnarray*}
  T_{sf}(n, 2) &=& (\sqrt{n}-2) \cdot 5 + T_{sf}(n, sqrt{n}) \\
  &=& 5\sqrt{n}-10 + 3 \\
  T(n, 2) &=& 2 \cdot (5\sqrt{n}-7)
\end{eqnarray*}

\begin{equation*}
  \boxed{T(n, 2) = O(\sqrt{n})}
\end{equation*}

\end{enumerate}

\section{Problem 6}

(10 points)

Write a function factorial(n) that uses recursion to calculate the
factorial of a given positive integer n. The factorial of a number n
is the product of all positive integers less than or equal to n. It is
denoted by n! and is defined as:

\[
n! = n * (n-1) * (n-2) * ... * 3 * 2 * 1
\]

By definition, $0! = 1$

Your function should:

Check if n is a non-negative integer. If not, the function should return an error message or handle the case appropriately.
Implement the base case: if n is 0, return 1.

\subsection*{Answer 6}

For full credit either return an error message or raise an exception
with an error message as in below:

\begin{verbatim}
    def factorial(n):
        if n < 0:
            raise Error("n must be nonnegative")
        if n ==1 or n == 0:
            return 1
        return n * factorial(n-1)
\end{verbatim}


\section{Problem 7}

(10 points)

Now we consider sorting. Given a $n$ numbers that all reside in the
$[1, 100]$ where $n >> 100$.  Write a code snippet to sort them in O(n)
time using python sorted. What is the time complexity? Now write a
sort that can sort these in $O(n)$ time. Hint: the fact that the range
is bounded and much less than n matters.

NOTE: Ignore the part using \verb|sorted| because I wrote it
incorrectly.  The intension was that the second sentence would read
``Write a code snippet to sort them using python sorted.''  Then the
subsequent sentendce ``What is the time complexity'' makes sense.


\subsection*{Answer 7}

\begin{verbatim}
    x = sorted(x)
\end{verbatim}

This has time complexity $O(n \log n)$.

The trick here is that the numbers are constrained to the range $[1,
  100]$.  As such we can create a table $cnts$ with 101 elements so
that the number of occurrences of any number in $i \in [1, 100]$ can
be tallied by incrementing $cnts[i]$.  


Here is a sort with linear time complexity:

\begin{verbatim}
    def linear_sort(x: list) -> list:
        n = len(x)              # (1) O(1) + 1 assign
        result = [None] * n     # (2) O(n)
        cnts = [0] * 101        # (3) O(101) = O(1)
        for i in x:             # (4) n * (5)
            cnts[i] += 1        # (5) 1 plus + 1 setitem
         
        k = 0                   # (6) 1 assign 
        for j in range(101):    # (7) 101 * (8 to 11)
            while cnts[j] > 0:  # (8) ? * (9 to 11)
                result[k] = j   # (9) 1 setitem
                k += 1          # (10) 1 increment
                cnts[j] -= 1    # (11) 1 minusequal + 1 setitem
        return result           # (12) O(1) pop activation record
\end{verbatim}

Let $T_{for}$ denote the time to execute lines (6) through (11).
Let $T_{while}$ denote the time to execute line (8) through (11).

\begin{eqnarray*}
  T(n) &=& O(1) + 1 + O(n) + O(101) + n \cdot (1 + 1) + T_{for}(n) + O(n) \\
  &=& O(1) + O(n) + 2n + T_{for}(n) \\
  &=& O(n) + T_{for}(n)
\end{eqnarray*}

The sum of all values in the \verb|cnts| array is equal to $n$ since
we performed one increment in step (5) for each value in $x$.  Each
iteration through the \verb|while| loop results in one value in the
\verb|cnts| array being decremented by 1.  As such, the while loop
will execute exactly $n$ times across all iterations through the
\verb|for| loop.  Thus, we can state

\begin{equation*}
  T_{while}(n) = n \cdot (1 + 1 + 1 + 1) = 4n
\end{equation*}

\begin{equation}
  T_{for}(n) = 101 \cdot (1 + 1) + 4n = 202 + 4n
\end{equation}

For large $n$, the $4n$ term dominates so

\begin{equation*}
  T(n) = O(n) + 202 + 4n
\end{equation*}

For large $n$, the 202 is ignorable.

\begin{equation*}
  \boxed{T(n) = O(n)}
\end{equation*}

\section{Problem 8} (10 points divided 5 and 5)

Given two sorted Python lists \verb|arr1| and \verb|arr2| of lengths
$n$ and $m$ respectively, write a function to find all the elements
that are common in both lists. Implement two solutions:

\begin{enumerate}[label=(\alph*)]

\item using python binary search (see python's \verb|bisect| module).  For each element in arr1, use binary search to check if it exists in arr2.  What is
  the time complexity as a function of $n$ and $m$? (5 points)

\subsection*{Answer 8(a)}

\begin{verbatim}
  common = []                    # (1) 1 assign
  for x in arr1:                 # (2) n * (3 to 5)
      i = bisect_left(arr2, x)   # (3) O(log(m)) 
      if arr2[i] == x:           # (4) 1 comp + 1 getitem
          common.append(x)       # (5) amortized O(1)
\end{verbatim}

\begin{equation*}
  T(n, m) = 1 + n \cdot \bigg(O(\log m) + 2
  + \text{ amortized } O(1)\bigg) 
\end{equation*}

\begin{equation*}
  \boxed{T(n, m) = O(n \log m)}
\end{equation*}

\item using a python dict (i.e., a hash table) Create a hash table
  from one array, then iterate through the other array to check for
  common elements.  What is the time complexity as a function of $n$
  and $m$? (5 points + 1 extra for ``average'')

\subsection*{Answer 8(b)}

It would make more sense to implement this using a
verb|set| rather than a \verb|dict|, but the problem calls for using
a dict.

\begin{verbatim}
    common = []              # (1) 1 assign
    d = {}                   # (2) 1 assign
    for x in arr1:           # (3) n1 * (4)
        d[x] = 1             # (4) average O(1) setitem
    for x in arr2:           # (5) n2  * (6 to 7)
        if x in d:           # (6) average O(1) \in
            common.append(x) # (7) amortized O(1)
\end{verbatim}

\begin{equation*}
  T(n) = 1 + 1 + n \cdot \text{ average } O(1)
         + m \cdot (\text{ average } O(1) + \text{ amortized } O(1))
\end{equation*}

``average $O(1)$'' is a looser constraint than ``amortized $O(1)$'' 
so we can ignore the ``amortized $O(1)$'' yielding

\begin{equation}
  \boxed{T(n) = \text{ average } O(n + m)}
\end{equation}

\end{enumerate}

\newpage

\section{Problem 9}

(10 points divided 5 and 5)

Given a list of $n$ integers.  We wish to find the $k^{th}$ smallest.

\begin{enumerate}[label=(\alph*)]
\item Write a function to find the the $k$th 
  smallest value using python's \verb|sorted|.  What is its time complexity
  as a function of $n$ and $k$? (5 points)

\subsection*{Answer 9(a)}

\begin{verbatim}
    def kth_smallest(x: list, k: int) -> int:
        return sorted(x)[k]  # O(n log n) sorted + 1 getitem
\end{verbatim}

\begin{equation}
  \boxed{T(n) = O(n \log n)}
\end{equation}

\item Write a function to find the $k$th smallest value using a binary
  min heap.  What is its time complexity as a function of $n$ and $k$?
  (5 points)

\subsection*{Answer 9(b)}

\begin{verbatim}
    from heapq import heapify, heappop

    def kth_smallest(x:list, k: int) -> int:
        """k uses the convention of starting
           from 0.  Therefore k=0 refers to the smallest."""
        if k < 0:
            raise IndexError("k must be nonnegative.")
        if len(x) == 0:
            raise ValueError("x must be a non-zero length list.")
        heapify(x)              # (1) O(n)
        y = None                # (2) O(1)
        for i in range(k+1):    # (3) (k+1) * (4)
            y = heappop(x)      # (4) log(n)
        return y                # (5) 1 pop activation record
\end{verbatim}

\begin{equation*}
  T(n, k) = O(n) + O(1) + (k+1) \log n + 1
\end{equation*}

\begin{equation}
  \boxed{T(n, k) = O(n + k \log n)}
\end{equation}

\end{enumerate}


\end{document}
