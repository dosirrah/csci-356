\documentclass{article} % Document class definition

% Preamble starts here
\usepackage{amsmath}
\usepackage{graphicx}
\usepackage{needspace}


\title{Homework 4 Answers}
% \date{Due date: 10/26/2023 at 11:59 PM}
\author{}

\begin{document}
\maketitle

% Redefine the label for the top-level items
\renewcommand{\labelenumi}{Problem \arabic{enumi}}

% Redefine the label for the second-level items
\renewcommand{\labelenumii}{\alph{enumii})}


\begin{enumerate}

\item (1 point each) is Problem 1 in Discussion Questions in 4.26.  of
  *Problem Solving with Algorithms and Data Structures using Python*.
  Put the answer in a text file named \verb|hw3_last_first/p1.txt|.
  Convert the following values to binary using “divide by 2.” Show the
  stack of remainders.


  \begin{enumerate}
  \item
      % needspace ensurese there is at least 5 lines worth of space after
      % the 17
      \needspace{5\baselineskip} % Enough room for first few lines of answer.
      17

      \begin{minipage}{4.5in}
        
        \vspace{0.1in}
        ANSWER
        
        \begin{eqnarray*}
          17 / 2 &=& 8 r 1 \text{ push 1, stack: [1]} \\
          8 / 2  &=& 4 r 0 \text{ push 0, stack: [1, 0]} \\
          4 / 2  &=& 4 r 0 \text{ push 0, stack: [1, 0, 0]} \\
          2 / 2  &=& 2 r 0 \text{ push 0, stack: [1, 0, 0, 0]} \\
          1 / 2  &=& 0 r 1 \text{ push 1, stack: [1, 0, 0, 0, 1]} 
        \end{eqnarray*}

        Now pop the stack until empty to obtain the binary number: 10001.

        It is not required that the answer be presented exact as above.

        Any notion of the stack [1, 0, 0, 0, 1] is adequate.
        \vspace{.1in}

      \end{minipage}

    
    \item
        \needspace{5\baselineskip} % Enough room for first few lines of answer.
        45
      
        \begin{minipage}{4.5in}
          \vspace{.1in}

          ANSWER

          \begin{eqnarray*}
            45 / 2 &=& 22 r 1 \text{ push 1, stack: [1]} \\
            22 / 2  &=& 11 r 0 \text{ push 0, stack: [1, 0]} \\
            11 / 2  &=& 5 r 1 \text{ push 1, stack: [1, 0, 1]} \\
            5 / 2  &=& 2 r 1 \text{ push 1, stack: [1, 0, 1, 1]} \\
            2 / 2  &=& 1 r 0 \text{ push 0, stack: [1, 0, 1, 1, 0]} \\
            1 / 2  &=& 1 r 1 \text{ push 1, stack: [1, 0, 1, 1, 0, 1]} 
          \end{eqnarray*}

          Now pop the stack until empty to obtain the binary number: 101101

          \vspace{.1in}
        \end{minipage}

    \item
      \needspace{5\baselineskip} % Enough room for first few lines of answer.
        96
      
        \begin{minipage}{4.5in}
          \vspace{.3in}

          ANSWER
          
          \begin{eqnarray*}
            96 / 2 &=& 48 r 0 \text{ push 0, stack: [0]} \\
            48 / 2 &=& 24 r 0 \text{ push 0, stack: [0, 0]} \\
            24 / 2 &=& 12 r 0 \text{ push 0, stack: [0, 0, 0]} \\
            12 / 2 &=& 6 r 0  \text{ push 0, stack: [0, 0, 0, 0]} \\
            6 / 2 &=& 3 r 0   \text{ push 0, stack: [0, 0, 0, 0, 0]} \\
            3 / 2 &=& 1 r 1   \text{ push 1, stack: [0, 0, 0, 0, 0, 1]} \\
            1 / 2 &=& 0 r 1   \text{ push 1, stack: [0, 0, 0, 0, 0, 1, 1]}
          \end{eqnarray*}

          Now pop the stack util empty to obtain the binary number 1100000.

          \vspace{.1in}
        \end{minipage}

  \end{enumerate}

\item (2 points)

  \begin{enumerate}
    \needspace{5\baselineskip} % Enough room for first few lines of answer.
    \item Create a \verb|LinkedList| class.  It must pass the
      \verb|hw3/p2/test_linked_list.py|.  The class MUST not use any
      Python built-in or standard library collection class, i.e., do
      not wrap a list or deque.
  
      \begin{minipage}{4.5in}
        \vspace{.1in}
        ANSWER

        I included an implementation of \verb|linked_list.py| in the
        repository.  The answer provided by the student must pass
        \verb|test_linked_list.py|.
        \vspace{.1in}
      \end{minipage}

  \item   Copy the Stack implementation found in the
  repository

  \begin{verbatim}
    https://git.cs.olemiss.edu/harrison/csci-356
  \end{verbatim}

  in \verb|lecture13and14/stack.py| into your homework directory
  \verb|hw3_last_first/p2|, rename the class \verb|LinkedListStack|
  and modify it so that it is implemented using your \verb|LinkedList|. It
  must pass the unit tests in the repository in the directory
  \verb|hw3/test_linked_stack.py|.  It MUST use your LinkedList.  the
  new Stack class MUST NOT use any Python built-in or standard library
  collection class, i.e., the \verb|LinkedListStack| class MUST not wrap a
  \verb|list| or \verb|deque|.

      \begin{minipage}{4.5in}
        \vspace{.1in}
        ANSWER

        I included an implementation of \verb|linked_list_stack.py| in
        the repository in \verb|hw3/|.  The answer provided by the
        student must pass \verb|test_linked_list.py|.
        \vspace{.1in}
      \end{minipage}

      
  \end{enumerate} 

\item (1 point each)
  Problem 3 in Discussion Questions in 4.26 of *Problem Solving with
  Algorithms and Data Structures using Python*. 

  Convert the infix expressions to postfix (use full parentheses).

  \begin{enumerate}\label{in2posta}
  \item
    \needspace{5\baselineskip}
    \verb|(A+B)*(C+D)*(E+F)|

    \begin{minipage}{4.5in}
      \vspace{.1in}
      ANSWER

      The book suggests the following procedure:

      \begin{quotation}
        
        1. Create an empty stack called opstack for keeping
        operators. Create an empty list for output.

        2. Convert the input infix string to a list by using the string
        method split.

        3. Scan the token list from left to right.

        4. If the token is an operand, append it to the end of the output list.

        5. If the token is a left parenthesis, push it on the opstack.

        6. If the token is a right parenthesis, pop the opstack until the
        corresponding left parenthesis is removed. Append each
        operator to the end of the output list.

        7. If the token is an operator, *, /, +, or -, push it on the
        opstack. However, first remove any operators already on the
        opstack that have higher or equal precedence and append them
        to the output list.

        When the input expression has been completely processed, check
        the opstack. Any operators still on the stack can be removed
        and appended to the end of the output list.
      \end{quotation}
    \end{minipage}

    \begin{minipage}{4.5in}
      \vspace{.1in}
      We would start with \verb|(A+B)*(C+D)*(E+F)|.

        \begin{verbatim}
          (A+B)*(C+D)*(E+F)  opstack = [], output=""
          A+B)*(C+D)*(E+F)   opstack = ['('], output=""
          +B)*(C+D)*(E+F)    opstack = ['('], output="A"
          B)*(C+D)*(E+F)     opstack = ['(', '+'], output="A"
          )*(C+D)*(E+F)      opstack = ['(', '+'], output="AB"
          *(C+D)*(E+F)       opstack = [], output="AB+"
          (C+D)*(E+F)        opstack = ['*'], output="AB+"
          C+D)*(E+F)         opstack = ['*', '('], output="AB+"
          +D)*(E+F)          opstack = ['*', '('], output="AB+C"
          
          D)*(E+F)           opstack = ['*', '(', '+'], output="AB+C"
          )*(E+F)            opstack = ['*', '(', '+'], output="AB+CD"
          *(E+F)             opstack = ['*'], output="AB+CD+"
          (E+F)              opstack = ['*'], output="AB+CD+*"
          E+F)               opstack = ['*', '('], output="AB+CD+*"
          +F)                opstack = ['*', '('], output="AB+CD+*E"
          F)                 opstack = ['*', '(', '+'], output="AB+CD+*E"
          )                  opstack = ['*', '(', '+'], output="AB+CD+*EF"
                             opstack = [], output="AB+CD+*EF+*"
        \end{verbatim}
         
        So the postfix generated by the algorithm presented in the book is
        given by
        
\begin{verbatim}
  AB+CD+*EF+*
\end{verbatim}

      \vspace{.1in}
    \end{minipage}

  \item
    \needspace{5\baselineskip} % Enough room for first few lines of answer.
    \verb|A+((B+C)*(D+E))|
    
    \begin{minipage}{4.5in}
      \vspace{.1in}
      ANSWER

      Using the analogous procedure given for the conversion for~\ref{in2post},
      we generate the following postfix:
      
          \verb|ABC+DE+*+|

      \vspace{.1in}
    \end{minipage}

      
  \item
    \needspace{5\baselineskip} 
    \verb|A*B*C*D+E+F|

    \begin{minipage}{4.5in}
      \vspace{.1in}
      ANSWER

      The procedure in the book would provide the following result:
      
          \verb|AB*C*D*E+F+|

      However, the following is an equivalent postfix notation for the given
      mathematical expression:

          \verb|ABCD***EF++|

      \vspace{.1in}
    \end{minipage}

  \end{enumerate}

%     **Problem 4**: (2 points) Use the `Queue` found in the source code
% repository in `hw3/p4/queue.py`, which is based on the code in the
% book in Listing 1 of Section 4.12

  
\end{enumerate}

\end{document}
